\documentclass[oneside,12pt]{scrartcl}
\usepackage[ngerman]{babel} %Deutsche Sprachunterstützung
\usepackage{scrpage2} %Kopf- und Fußzeilen
\usepackage[utf8]{inputenc} %Umlaute
\usepackage{tabularx}
\usepackage{eurosym}
\pagestyle{scrheadings}

\newcommand{\code}[1]{\texttt{#1}}
\newenvironment{codeblock}{\ttfamily{}{}}

\begin{document}
\setlength{\parindent}{0pt} %Dummes Absatz-Eingerücke abstellen
\cofoot{}
\rofoot{\pagemark}

\begin{center}
\Huge{CLog Documentation} \par
\Large{HHU-Programmierpraktikum SS2016 Projekt 5}
\end{center}

\section{Beschreibung}
Der Nutzer startet das Programm im Terminal mit dem Aufruf \code{java Clog}. Beim Starten des Programms landet der Benutzer direkt im Hauptmenü des Programmes. Ihm werden die vorgegebenen Auswahlmöglichkeiten geboten, sowie eine zusätzliche Auswahlmöglichkeit \code{6}, alle Datensätze auszugeben:

\begin{verbatim}
Hauptmenü:
1) Clog-Eintrag erzeugen
2) Clog ausgeben
3) Clog laden
4) Clog speichern
5) Programm beenden
6) Alle Datensätze ausgeben
\end{verbatim}

\subsection{Menüpunkt 1: \code{Clog-Eintrag erzeugen}}

\end{document}